% This is a simple sample document.  For more complicated documents take a look in the exercise tab. Note that everything that comes after a % symbol is treated as comment and ignored when the code is compiled.

\documentclass{article} % \documentclass{} is the first command in any LaTeX code.  It is used to define what kind of document you are creating such as an article or a book, and begins the document preamble
\usepackage{listings}
\usepackage{xcolor}
\usepackage{amsmath} % \usepackage is a command that allows you to add functionality to your LaTeX code
\definecolor{codegreen}{rgb}{0,0.6,0}
\definecolor{codegray}{rgb}{0.5,0.5,0.5}
\definecolor{codepurple}{rgb}{0.58,0,0.82}
\definecolor{backcolour}{rgb}{0.95,0.95,0.92}
\title{IPSec Debian-Windows Debian része} % Sets article title
\author{Szőke Attila} % Sets authors name
\date{\today} % Sets date for date compiled

% The preamble ends with the command \begin{document}
\begin{document} % All begin commands must be paired with an end command somewhere
    \maketitle % creates title using information in preamble (title, author, date)


\vspace{5mm}
    \section*{Alagút interfészek} % creates a section 

    Ez az első verzió, amely egy fölösleges GRE interfészt használ, a következőben ez nem lesz benne.
    
    A Debian IP címe 10.10.10.1, a Windows-é 10.10.10.2. Az első alagút interfész a 10.20.20.0/24-es hálózaton történik, ezeknek elég lenne /30-asnak lenni.
    
    A második erre épül, a 10.30.30.0/24-esen megy, a harmadik pedig a 10.40.40.0/24-esen. \\

    Győződjünk meg róla, hogy be van-e töltve az ip\_gre modul:

    \begin{itemize}
	    \item[\#] lsmod | grep gre
    \end{itemize}

    Amennyiben nincs kimenete az előző parancsnak, töltsük be:

    \begin{itemize}
	    \item[\#] modprobe ip\_gre
    \end{itemize}

    Ehhez a parancsok a következők:
    \begin{itemize}
	    \item[\#] ip tunnel add tun0 mode gre remote 10.10.10.2 local 10.10.10.1 ttl 255
	    \item[\#] ip link set tun0 up
	    \item[\#] ip addr add 10.20.20.1/24 dev tun0
	    \item[\#] ip tunnel add tun1 mode gre remote 10.20.20.2 local 10.20.20.1 ttl 255
	    \item[\#] ip link set tun1 up
	    \item[\#] ip addr add 10.30.30.1/24 dev tun1
	    \item[\#] ip tunnel add tun2 mode gre remote 10.30.30.2 local 10.30.30.1 ttl 255
	    \item[\#] ip link set tun2 up
	    \item[\#] ip addr add 10.40.40.1/24 dev tun2
\end{itemize}




    \section*{Strongswan} % creates a section 

    Telepítsük fel a strongswan-t:

    \begin{itemize}
	    \item[\#] apt install strongswan
    \end{itemize}

    Írjuk át a \lstinline|/etc/ipsec.conf|-ot:

    \begin{lstlisting}
config setup
        charondebug="all"
        uniqueids=yes
conn devgateway-to-prodgateway
        type=tunnel
        auto=add
        auto=start
        keyexchange=ikev2
        authby=secret
        left=10.20.20.1
        leftsubnet=10.30.30.0/24
        right=10.20.20.2
        rightsourceip=10.20.20.2
        rightsubnet=10.30.30.0/24
        ike=aes256-sha1-modp1024!
        esp=aes256-sha1!
        aggressive=no
        keyingtries=%forever
        ikelifetime=28800s
        lifetime=3600s
        dpddelay=30s
        dpdtimeout=120s
        dpdaction=restart
    \end{lstlisting}

    Írjuk át a \lstinline|/etc/ipsec.secrets|-et: \\
    \lstinline|%any %any : PSK ``Skill39''| \\


    Indítsuk (újra) a strongswan-t: \\
    \lstinline|systemctl restart strongswan-starter|

\end{document}

